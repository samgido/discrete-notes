\author{Samuel Gido}
\title{Week 2 Discrete}
\date{Spring 2024}

\documentclass[12pt, letterpaper]{article}

\usepackage{amsmath}
\usepackage{fullpage}
\usepackage{setspace}
\onehalfspacing

\begin{document} 
\maketitle

\section*{Combinations of Sets}

Let \(S\) be a set of \(n\) elements, A \textit{combination} of set \(S\) is an unordered selection 
of \(r\) elements in \(S\). The result is a subset, \(A\), of \(S\). The number of \(r\)-subset or \(r\)-combination
of \(S\) with \(n\) elements is denoted by \(\binom{n}{r}\) and the formula is: 

\begin{center}
  \(\binom{n}{r} = \frac{n!}{r!(n-r)!}\)
\end{center}

\noindent For example, if \(S = \{a, b, c, d\}\), there are 3-subsets of \(S\): 

\begin{center}
  \(
    \{a, b, c\}, \{a, b, d\}, \{a, c, d\}, \{b, c, d\}
  \)
\end{center}

\noindent Obvious rules for combinations:

\begin{center}
  \( \binom{n}{r} = 0 \) if \(r > n\), \\
  \( \binom{0}{r} = 0 \) if \(r > 0\) \\
  And, \\
  \( \binom{n}{0} = 1,\, \binom{n}{1}=n,\, \binom{n}{n} = 1\)
\end{center}

\subsection*{Properties of Combinations}

\begin{center}
  \( \binom{n-1}{k} + \binom{n-1}{k-1} = \binom{n}{k} \)
\end{center}

\pagebreak
\section*{Combinations of Mulit-Sets}
Multisets are sets that hold multiple copies of the same element. For example, 
the multiset \(\{a, a, b, c, c, c\}\), also written as \(\{2a, b ,3c\}\) \\

\noindent \(r\)-permutations of a multiset \(S\) is an ordered arrangment of \(r\) elements of \(S\). 
While there are many formulas for the amount of permutations of a multiset, one existst that gives 
the number of full permutations of \(S\): 

\begin{center}
  \(\frac{n!}{n_1! \cdot n_2! \cdot \dots \cdot n_k! }\)
\end{center}


\end{document}