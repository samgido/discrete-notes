\author{Samuel Gido}
\title{Week 2 Discrete}
\date{Spring 2024}

\documentclass[12pt, letterpaper]{article}

\usepackage{amsmath}
\usepackage{fullpage}
\usepackage{setspace}
\onehalfspacing

\begin{document} 
\maketitle

\section*{Combinations of Sets}

Let \(S\) be a set of \(n\) elements, A \textit{combination} of set \(S\) is an unordered selection 
of \(r\) elements in \(S\). The result is a subset, \(A\), of \(S\). An \(r\)-subset or \(r\)-combination
of \(S\) is denoted by \(\binom{n}{r}\). \\ 

\noindent For example, if \(S = \{a, b, c, d\}\), the four 3-subsets of \(S\) are: 

\begin{center}
  \(
    \{a, b, c\}, \{a, b, d\}, \{a, c, d\}, \{b, c, d\}
  \)
\end{center}

\noindent Obvious rules for combinations:

\begin{center}
  \( \binom{n}{r} = 0 \) if \(r > n\), \\
  \( \binom{0}{r} = 0 \) if \(r > 0\) \\
  And, \\
  \( \binom{n}{0} = 1,\, \binom{n}{1}=n,\, \binom{n}{n} = 1\)
\end{center}

\end{document}