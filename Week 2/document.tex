\author{Samuel Gido}
\title{Week 2 Discrete}
\date{Spring 2024}

\documentclass[12pt, letterpaper]{article}

\usepackage{amsmath}
\usepackage{fullpage}
\usepackage{setspace}
\onehalfspacing

\begin{document} 
\maketitle

\section*{Combinations of Sets}

Let \(S\) be a set of \(n\) elements, A \textit{combination} of set \(S\) is an unordered selection 
of \(r\) elements in \(S\). The result is a subset, \(A\), of \(S\). The number of \(r\)-subset or \(r\)-combination
of \(S\) with \(n\) elements is denoted by \(\binom{n}{r}\) and the formula is: 

\begin{center}
  \(\binom{n}{r} = \frac{n!}{r!(n-r)!}\)
\end{center}

\noindent For example, if \(S = \{a, b, c, d\}\), there are 3-subsets of \(S\): 

\begin{center}
  \(
    \{a, b, c\}, \{a, b, d\}, \{a, c, d\}, \{b, c, d\}
  \)
\end{center}

\noindent Combinations are different from permuations because when finding combinations, the results are sets, and the order of a elements in a set do not matter. This is why \(\{a, b, c\}\) exists but \(\{b, a, c\}\) doesn't. \\

\noindent Obvious rules for combinations:

\begin{center}
  \( \binom{n}{r} = 0 \) if \(r > n\), \\
  \( \binom{0}{r} = 0 \) if \(r > 0\) \\
  And, \\
  \( \binom{n}{0} = 1,\, \binom{n}{1}=n,\, \binom{n}{n} = 1\)
\end{center}

\subsection*{Properties of Combinations}

\begin{center}
  \( \binom{n-1}{k} + \binom{n-1}{k-1} = \binom{n}{k} \)
\end{center}

\pagebreak


\section*{Multi-Sets}
Multisets are sets that hold multiple copies of the same element. For example, 
the multiset \(\{a, a, b, c, c, c\}\), also written as \(\{2a, b ,3c\}\). The 
'coefficients' attached to each element is sometimes called a repetition number. 

\section*{Permuations of Multi-Sets}

\noindent 
An \(r\)-permutations of a multiset \(S\) is an ordered arrangment of \(r\) elements of \(S\). 
There are 2 major theorems for finding the number of permuations of a multiset,
one gives the number of permuations when the repetition numbers of \textit{all} elements are infinite, 
and the other gives the number of permuations when the repetition numbers of all elements are finite. \\

\noindent
\textbf{Theorem} Let \(S\) be a multiset with \(k\) types of objects, where each object has an infinite repetition number.
The number of \(r\)-permuations of \(S\) is \(k^r\). \\

\noindent 
\textbf{Theorem} Let \(S\) be a multiset with \(k\) different types of objects, each with finite repetition numbers,
\(n_1,n_2,\dots, n_k\) respectively. Let the size of \(S\) be \(n = n_1 + n_2 + \dots + n_k\).
The number of permuations of \(S\) equals 

\[
  \frac{n!}{n_1! n_2! \dots n_k!}
\]

From this theorem, another interpretation occurs when partitioning a set of objects into parts of 
prescribed size and where the parts have labels to them. This is best shown with an example. \\ 

\noindent
\textbf{Example} Consider a set of four objects \(\{a, b, c, d\}\) that needs to be partitioned into two sets each of size 2.
If each part is not labeled, then there are 3 different ways of partitioning. 

\[
  \{a, b\}, \{c, d\}; \{a, c\}, \{b, d\}; \{a, d\}, \{b, c\};
\]

Now suppose that each part is labeled, maybe where each part is a box and the label is the color of the box. In this case,
The number of permutations is now 6. This can be generalized into the following theorem.

\noindent 
\textbf{Theorem} Let \(n\) be a positive integer and let \(n_1, n_2, \dots n_k\) be positive integers with
\(n = n_1 + n_2 + \dots + n_k \). The number of ways to partition a set of \(n\) objects into \(k\) \textit{labeled} boxes 
where box 1 contains \(n_1\) objects, box 2 contains \(n_2\) objects, \(\dots\) box \(k\) contains \(n_k\) objects equals. 

\[
  \frac{n!}{n_1! n_2! \dots n_k!}
\]

If the boxes are not labeled and \(n_1 = n_2 = \dots = n_k \), then the number of partitions equals

\[
  \frac{n!}{k! n_1! n_2! \dots n_k!}
\]

\section*{Combinations of Multisets}

If \(S\) is a multiset, then an \(r\)-combination of \(S\) is an unordered selection of \(r\) of the objects of \(S\).
Thus, an \(r\)-combination of \(S\) is itself a multiset, or for short, an \(r\)-submultiset. If \(S\)
has \(n\) objects, then there is only one \(n\)-combination of \(S\), namely \(S\) itself. Similarly to permuations of multisets, 
we first count the number of \(r\)-combinations of a multiset whose repetition numbers are all infinte. \\

\noindent
\textbf{Theorem} Let \(S\) be a multiset with objects of \(k\) types, each with an infinite repetition number.
Then the number of \(r\)-combinations of \(S\) equals 

\[
  \Big( \frac{r + k - 1}{r} \Big) = \Big( \frac{r + k - 1}{k - 1} \Big) = \frac{(r + k - 1)!}{r! (k - 1)!}
\]

\end{document}