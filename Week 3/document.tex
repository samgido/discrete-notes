\author{Samuel Gido}
\title{Week 3 Discrete Math Notes}
\date{Spring 2024}

\documentclass[12pt, letterpaper]{article}

\usepackage{amsmath}
\usepackage{fullpage}
\usepackage{setspace}
\onehalfspacing

\begin{document} 
\maketitle

\section*{Pigeonhole Principle}

\subsection*{Simple Form}

If \(n+ 1\) objects are distributed into \(n\) boxes, then at least one box contains two or more of the objects. 

\subsection*{Stronger Form}

Let \(q_1,q_2, \dots, q_n\) be positive integers. If
  \begin{center}
    \((q_1 + q_2+ \dots + q_n) - n + 1\)
  \end{center}
Objects are distributed into \(n\) boxes, then either the first box contains at least \(q_1\) objects, or the second box contains at least \(q_2\) boxes, \dots, or the \(n\)th box contains at least \(q_n\) objects. \indent

\noindent There is special case, where \(q_1, q_2, \dots, q_n\) are all equal to come integer \(r\), this is called the corollary and it is where the simple form is derived from. \\

\noindent \textbf{Corollary}:

\begin{center}
  Let \(n\) and \(r\) be positive integers. If \(n(r-1) + 1\) objects are distributed into \(n\) boxes, then at least one of the boxes contains \(r\) or more objects. 
\end{center}

\end{document}